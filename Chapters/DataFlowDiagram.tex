\section{Data Flow Diagram}
\noindent In questo capitolo si parlerà delle specifiche imposte sui blocchi operazionali.
Si andrà a confrontare gli approcci "As Soon As Possible" \(ASAP\) e "As Late As Possible" \(ALAP\).
Infine verrà illustrato l'approccio che è stato utilizzato per ottimizzare le tempistiche dell'algoritmo.

\subsection{Specifiche sui blocchi operazionali}
\noindent In questo Progetto si supponeva di poter utilizzare per ciascuna Butterfly un unico blocco moltiplicatore.
In grado di poter svolgere sia l'operazione di moltiplicazione tra due numeri in ingresso sia come un moltiplicatore per 2 di un dato in ingresso.
Le due operzioni sono selezionabili attraverso un segnale di controllo esterno al blocco operatore.
Si è supposto che il moltiplicatore avesse due livelli di pipeline, ovvero che il risultato fosse disponibile al registro di uscita dopo 3 colpi di clock.
Mentre l'operazione di shift (moltiplicazione per 2) avesse un livello di pipeline, dunque l'uscita sarebbe disponibile dopo 2 colpi di clock.
Il blocco moltiplicatore è stato rappresentato in \textit{Fig. \ref{fig: blockdataflowdiagram}} con il blocco \(verde\) e l'operazione di shift è stata rappresentata con il blocco \(viola\).

Si è supposto di avere a disposizione un singolo elemento per le operazioni di somma e uno per le sottrazioni.
I blocchi sommatore e sottrattore hanno ciascuno un livello di pipeline e sono rappresentati rispettivamente dal blocco \(rosso\) e \(blu\).

Al fine di rappresentare anche l'operzione di ROM Rounding presente alla fine dell'algoritmo è stato deciso di impiegare un colpo di clock per l'operzione di arrotondamento (blocco \(azzurro\)) e 
utilizzare successivamente un registro controllato esternamente per mantenere in vita le variabili di uscita della butterfly.

\begin{figure}[H]
    \centering
    \includegraphics[width=0.75\textwidth]{Foto/DataFlowDiagram/BlocchiDataFlowDiagram.jpg}
    \caption{Blocchi elementari del data flow diagram}
    \label{fig: blockdataflowdiagram}
\end{figure}

\subsection{Approccio ASAP}
\noindent L'approccio "As Soon As Possible", è un approccio che predilige lo svolgimento delle operazioni non appena si ha disponibilità.
Come si può notare in \textit{Fig. \ref{fig: asapdataflow}} sarebbero necessari 6 blocchi moltiplicatori e 2 blocchi sommatore e sottrattore.
Questo tipo di schema non ci permette di rispettare dunque la specifica sul numero di blocchi operazionali.
\begin{figure}[H]
    \centering
    \includegraphics[width=0.75\textwidth]{Foto/DataFlowDiagram/asap_DataFlow.jpg}
    \caption{Data Flow Diagram con pproccio ASAP}
    \label{fig: asapdataflow}
\end{figure}

\subsection{Approccio ALAP}
\noindent L'approccio "As Late As Possible", questo approccio predilige lo svolgimento delle operazioni il più tardi possibile.
Lo schema riportato in \textit{Fig. \ref{fig: alapdataflow}} mostra come il numero di blocchi operazionali richiesti è inferiore rispetto all'approccio ASAP,
infatti vengono utilizzati 2 elementi per ciascun operzione di moltiplicazione, somma, differenza e arrotondamento.
Anche in questo caso però non viene rispettato il limite numerico di 1 blocco per Butterfly.
Verrà dunque studiato un approccio che ci permetta di rimanere entro le specifiche numeriche.


\begin{figure}[H]
    \centering
    \includegraphics[width=0.75\textwidth]{Foto/DataFlowDiagram/alap_DataFlow.jpg}
    \caption{Data Flow Diagram con approccio ALAP}
    \label{fig: alapdataflow}
\end{figure}

\subsection{Approccio scelto}
\noindent Per ottimizzare le tempistiche dell'algoritmo avendo delle restrizioni sul numero di operatori si è optato per il Data Flow Diagram illustrato in \textit{Fig. \ref{fig: dataflowdiagram}}.
Questo approccio è stato studiato per l'esecuzione dell'algoritmo in modo da avere ad ogni stadio un unico blocco operazionale per tipo di operazione,
rientrando nelle specifiche imposte sul progetto.

\begin{figure}[H]
    \centering
    \includegraphics[width=0.75\textwidth]{Foto/DataFlowDiagram/DataFlowDiagram.jpg}
    \caption{Data flow diagram ottimizzato}
    \label{fig: dataflowdiagram}
\end{figure}

\subsection{Tempo di vita delle variabili}
\noindent In \textit{Fig. \ref{fig: tempodivita}} viene illustrato il tempo di vita delle variabili derivato dal Data Flow Diagram che è stato utilizzato.
Questo ci è utile per capire quando un segnale deve essere conservato in un determinato registro, da questo possiamo ricavare i segnali, derivanti dalla control unit, per controllare i registri.
Nel nostro schema gli unici registri che necessitano un segnale di controllo esterno sono i registri di ingresso e uscita della Butterfly.
I registri posizionati tra i blocchi operazionali, dato che sono "vivi" per un solo colpo di clock non hanno bisogno di essere controllati.

\begin{figure}[H]
    \centering
    \includegraphics[width=0.75\textwidth]{Foto/DataFlowDiagram/tempo_di_vita_variabili.jpg}
    \caption{Tempo di vita delle variabili}
    \label{fig: tempodivita}
\end{figure}

\pagebreak