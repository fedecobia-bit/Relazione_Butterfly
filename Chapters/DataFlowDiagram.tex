\section{Data Flow Diagram}
\noindent In questo capitolo si parlerà delle specifiche imposte sui blocchi logici.
Si andranno a confrontare gli approcci "As Soon As Possible" \(ASAP\) e "As Late As Possible" \(ALAP\).
Infine verrà illustrato l'approccio che è stato utilizzato per ottimizzare le tempistiche dell'algoritmo.

\subsection{Specifiche sui blocchi operazionali}

\noindent In questo Progetto si supponeva di poter utilizzare per ciascuna Butterfly un unico blocco moltiplicatore.
Questo blocco è in grado di poter svolgere sia la moltiplicazione tra due numeri in ingresso sia lo shift (moltiplicazione per 2).
Le due operazioni sono selezionabili attraverso un segnale di controllo esterno al blocco operatore che verrà fornito dalla
Contro Unit (CU).
Si è supposto che il moltiplicatore avesse due livelli di pipeline, ovvero che il risultato fosse disponibile al registro
di uscita dopo 3 colpi di clock.
L'operazione di shift, al contrario, ha un singolo livello di pipeline, dunque l'uscita è disponibile dopo 2 colpi di clock.
Il blocco moltiplicatore è stato rappresentato in \textit{Fig. \ref{fig: blockdataflowdiagram}} con il blocco \(verde\) e
l'operazione di shift è stata rappresentata con il blocco \(viola\).

Si è supposto di avere a disposizione un singolo elemento per le operazioni di somma e uno per le sottrazioni.
I blocchi sommatore e sottrattore hanno ciascuno un livello di pipeline e sono rappresentati rispettivamente
dal blocco \(rosso\) e \(blu\).

Al fine di rappresentare anche l'operazione di ROM Rounding presente alla fine dell'algoritmo è stato deciso di impiegare
un colpo di clock per l'operazione di arrotondamento (blocco \(azzurro\)) e utilizzare successivamente un registro
controllato esternamente per mantenere in vita le variabili di uscita della butterfly.

\begin{figure}[H]
    \centering
    \includegraphics[width=0.75\textwidth]{Foto/DataFlowDiagram/BlocchiDataFlowDiagram.jpg}
    \caption{Blocchi elementari del data flow diagram}
    \label{fig: blockdataflowdiagram}
\end{figure}

\subsection{Approccio ASAP}
\noindent L'approccio "As Soon As Possible", è un approccio che predilige lo svolgimento delle operazioni non appena si
ha disponibilità di blocchi logici.
Come si può notare in \textit{Fig. \ref{fig: asapdataflow}} sarebbero necessari 6 blocchi moltiplicatori e 2 blocchi
sommatore e sottrattore.
Andando a confrontare le richieste di questa metodologia con le specifiche di progetto risulta subito evidente che questo approccio
non è applicabile.

\begin{figure}[H]
    \centering
    \includegraphics[width=0.75\textwidth]{Foto/DataFlowDiagram/asap_DataFlow.jpg}
    \caption{Data Flow Diagram con pproccio ASAP}
    \label{fig: asapdataflow}
\end{figure}

\subsection{Approccio ALAP}
\noindent Questo approccio predilige lo svolgimento delle operazioni il più tardi possibile.
Lo schema riportato in \textit{Fig. \ref{fig: alapdataflow}} mostra come il numero di blocchi operazionali richiesti sia
inferiore rispetto all'approccio ASAP.
Vengono infatti utilizzati 2 elementi per ciascuna operazione di moltiplicazione, somma, differenza e arrotondamento.
Anche in questo caso però non viene rispettato il limite numerico di 1 blocco moltiplicatore per Butterfly.
\'{E} stato dunque studiato un approccio che ci permettesse di rimanere entro le specifiche richieste dal progetto.


\begin{figure}[H]
    \centering
    \includegraphics[width=0.75\textwidth]{Foto/DataFlowDiagram/alap_DataFlow.jpg}
    \caption{Data Flow Diagram con approccio ALAP}
    \label{fig: alapdataflow}
\end{figure}

\subsection{Approccio scelto}
\noindent Volendo ottimizzare al massimo il sistema pur rimanendo all'interno delle specifiche imposte si è optato per 
il Data Flow Diagram illustrato in \textit{Fig. \ref{fig: dataflowdiagram}}.
Questo approccio è stato studiato per l'esecuzione dell'algoritmo in modo da avere ad ogni stadio un unico blocco logico 
per tipo di operazione.

\begin{figure}[H]
    \centering
    \includegraphics[width=0.75\textwidth]{Foto/DataFlowDiagram/DataFlowDiagram.jpg}
    \caption{Data flow diagram ottimizzato}
    \label{fig: dataflowdiagram}
\end{figure}

\subsection{Tempo di vita delle variabili}
\noindent In \textit{Fig. \ref{fig: tempodivita}} viene illustrato il tempo di vita delle variabili derivato dal Data Flow Diagram
che è stato scelto.
Questo studio risulta utile per capire quanto una variabile deve rimanere in vita e, di conseguenza, quanti registri sono necessari
per gestire tutte le variabili all'interno del sistema.
Andando a studiare qual è il numero massimo richiesto di registri si riduce il costo e l'area occupata su Silicio.
Inoltre, si facilita lo studio di quali e quanti segnali sono richiesti dalla CU per controllare il flusso dei dati.

In questo caso però, per come è stato progettato il sistema, gli unici registri che necessitano un segnale di controllo
esterno sono i registri di ingresso e d'uscita della Butterfly.
I registri posizionati tra i blocchi logici, dato che sono "vivi" per un solo colpo di clock, non necessitano di segnali di
controllo.

\begin{figure}[H]
    \centering
    \includegraphics[width=0.75\textwidth]{Foto/DataFlowDiagram/tempo_di_vita_variabili.jpg}
    \caption{Tempo di vita delle variabili}
    \label{fig: tempodivita}
\end{figure}

\pagebreak