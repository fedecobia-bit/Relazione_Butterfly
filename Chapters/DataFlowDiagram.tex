\section{Data Flow Diagram}
\noindent In questo capitolo verrà illustrato l'approccio che è stato scelto per il data flow diagram.
Andando prima a confrontare gli approcci "As Soon As Possible" \(ASAP\) e "As Late As Possible" \(ALAP\).
Infine verranno illustrati i blocchi operazionali prosenti nelle specifiche di progetto e l'approccio che è stato utilizzato per ottimizzare le tempistiche dell'algoritmo.

\subsection{Specifiche blocchi operazionali}

\subsection{Approccio ASAP}
\noindent L'approccio "As Soon As Possible", è un approccio che predilige lo svolgimento delle operazioni non appena si ha disponibilità.

\subsection{Approccio ALAP}

\begin{figure}[H]
    \centering
    \includegraphics[width=0.75\textwidth]{Foto/DataFlowDiagram/BlocchiDataFlowDiagram.jpg}
    \caption{Blocchi elementari del data flow diagram}
    \label{fig: blockfdataflowdiagram}
\end{figure}
