\section{Data Flow Diagram}
\noindent In questo capitolo si parlerà delle specifiche imposte sui blocchi operazionali.
Si andrà a confrontare gli approcci "As Soon As Possible" \(ASAP\) e "As Late As Possible" \(ALAP\).
Infine verrà illustrato l'approccio che è stato utilizzato per ottimizzare le tempistiche dell'algoritmo.

\subsection{Specifiche sui blocchi operazionali}
\noindent In questo Progetto si supponeva di poter utilizzare per ciascuna Butterfly un unico blocco moltiplicatore.
In grado di poter svolgere sia l'operazione di moltiplicazione tra due numeri in ingresso sia come un moltiplicatore per 2 di un dato in ingresso.
Le due operzioni erano selezionabili attraverso un segnale di controllo esterno al blocco operatore.
Si è supposto che il moltiplicatore avesse due livelli di pipeline, ovvero che il risultato fosse disponibile al registro di uscita dopo 3 colpi di clock.
Mentre l'operazione di shift (moltiplicazione per 2) avesse un livello di pipeline e quindi l'uscita sarebbe disponibile dopo 2 colpi di clock.
Il blocco moltiplicatore è stato rappresentato in \ref{fig: blockdataflowdiagram} con il blocco \(verde\) e mentre l'operazione di shift è stata rappresentata con il blocco \(viola\).

I blocchi sommatore e sottrattore avevano entrambi un livello di pipeline e sono rappresentati rispettivamente dal blocco \(rosso\) e \(blu\).

Al fine di rappresentare anche l'operzione di ROM Rounding presente alla fine dell'algoritmo è stato deciso di impiegare un colpo di clock per l'operzione di arrotondamento (blocco \(azzurro\)) e 
utilizzare successivamente un registro controllato esternamente per mantenere in vita le variabili di uscita della butterfly.




\begin{figure}[H]
    \centering
    \includegraphics[width=0.75\textwidth]{Foto/DataFlowDiagram/BlocchiDataFlowDiagram.jpg}
    \caption{Blocchi elementari del data flow diagram}
    \label{fig: blockdataflowdiagram}
\end{figure}

\subsection{Approccio ASAP}
\noindent L'approccio "As Soon As Possible", è un approccio che predilige lo svolgimento delle operazioni non appena si ha disponibilità.
da continuare.
inserire foto.

\subsection{Approccio ALAP}
\noindent L'approccio "As Late As Possible", questo approccio predilige lo svolgimento delle operazioni il più tardi possibile.
da continuare.
inserire foto.

\subsection{Approccio scelto}
Per ottimizzare le tempistiche dell'algoritmo avendo delle restrizioni sul numero di operatori si è optato per il data flow diagram illustrato in \ref{fig: dataflowdiagram}

\begin{figure}[H]
    \centering
    \includegraphics[width=0.75\textwidth]{Foto/DataFlowDiagram/DataFlowDiagram.jpg}
    \caption{Data flow diagram}
    \label{fig: dataflowdiagram}
\end{figure}


\pagebreak