\section{Appendice - File VHDL}
\label{cap: Appendice}

\noindent Come richiesto nelle specifiche del progetto si riportano di seguito i listati VHDL utilizzati per sviluppare/testare
il progetto.
Si è inserito il nome e la matricola di ogni componente del gruppo nei vari listati.


% Per facilitare il tutto questo è il comando da usare per scrivere il codice:
% 1) Lasciare all'inizio il nome dei componenti del gruppo
% 2) Cambiare il nome nella caption ___ se è un test bench chiamarlo TB***
% 								                   |__ se è un normale chiamarlo col suo nome
%
% Per utilizzare il codice basta copiare e incollare la roba dentro \begin{comment}\end{comment}
\begin{comment}

\begin{lstlisting}[caption={TBROM}]
-- Federico Cobianchi - 332753
-- Onice Mazzi - 359754
-- Antonio Telmon - 353781

\end{lstlisting}

\end{comment}

\subsection{Sommatore}

%\begin{lstlisting}[caption={Sommatore}]
% \input{Code/Sommatore.tex}
%\end{lstlisting}

\lstinputlisting[
  language=VHDL,
  caption={Sommatore}
]{Code/Components/bfly_adder.vhd}

\subsection{MUX}

\subsubsection{MUX 2}

\lstinputlisting[
  language=VHDL,
  caption={MUX 2}
]{Code/Components/mux_2.vhd}

\subsubsection{MUX 3}

\lstinputlisting[
  language=VHDL,
  caption={MUX 3}
]{Code/Components/mux_3.vhd}

\subsection{Sottrattore}

\lstinputlisting[
  language=VHDL,
  caption={Sottrattore}
]{Code/Components/bfly_subtractor.vhd}

\subsection{Moltiplicatore/Shifter}

\lstinputlisting[
  language=VHDL,
  caption={Moltiplicatore/shifter}
]{Code/Components/bfly_multiplier.vhd}

\subsection{ROM rounding}

\subsubsection{Blocco ROM rounding}

\lstinputlisting[
  language=VHDL,
  caption={Blocco intero adibito al ROM rounding}
]{Code/Components/Rounding_single_clock.vhd}

\subsubsection{Test Bench del ROM rounding}

\lstinputlisting[
  language=VHDL,
  caption={Test Bench del ROM rounding}
]{Code/Test Bench/Testbench_ROUNDING.vhd}

\subsubsection{ROM}

\lstinputlisting[
  language=VHDL,
  caption={ROM}
]{Code/Components/ROM.vhd}

\subsubsection{Test Bench della ROM}

\lstinputlisting[
  language=VHDL,
  caption={Test Bench della ROM}
]{Code/Test Bench/Testbench_ROM.vhd}

\subsection{Datapath}

\lstinputlisting[
  language=VHDL,
  caption={Datapath}
]{Code/Components/bfly_datapath.vhd}

\subsection{Control Unit}

\subsubsection{Datapath}
\label{listing: control_unit_datapath}

\lstinputlisting[
  language=VHDL,
  caption={Control Unit datapath}
]{Code/Components/bfly_CU_datapath.vhd}

\subsubsection{ROM}

\lstinputlisting[
  language=VHDL,
  caption={Control Unit ROM}
]{Code/Components/bfly_CU_ROM.vhd}

\subsubsection{PLA}

\lstinputlisting[
  language=VHDL,
  caption={Control Unit PLA}
]{Code/Components/bfly_CU_late_status_PLA.vhd}

\subsubsection{Test Bench della Control Unit}

\lstinputlisting[
  language=VHDL,
  caption={Test Bench della Control Unit}
]{Code/Test Bench/tb_CU.vhd}

\subsection{FFT}

\lstinputlisting[
  language=VHDL,
  caption={FFT}
]{Code/Components/bfly_top_entity.vhd}

\subsubsection{Test Bench della Butterfly singola}

\lstinputlisting[
  language=VHDL,
  caption={Test Bench della FFT}
]{Code/Test Bench/TB_FFT.vhd}