\section{Butterfly}

\noindent Dopo aver descritto, tramite linguaggio VHDL, i vari blocchi precedentemente descritti si è passati alla creazione di un
blocco butterfly singolo per un test intermedio e per accertarsi che tutto funzionasse in modo sinergico.
I listati possono essere consultati nel \textit{Cap. \ref{cap: Appendice}}.

Questo test è risultato importante perchè ha permesso di correggere alcuni piccoli errori che non erano stati individuati durante
i test precedenti.
Grazie a queste modifiche il singolo blocco butterfly ha funzionato correttamente durante i successivi test e ciò a permesso di 
proseguire nell'implementazione della FFT 16x16 senza doversi preoccupare di possibili errori commessi in precedenza.\\

\noindent Come si vede da \textit{Fig. \ref{fig: Butterfly_schematic}} la butterfly singola riceve in ingresso segnali con
parallelismo 24 bit e il segnale SF\_2H\_1L per gestire lo shift.
In uscita presenta un bus con parallelismo 24 bit.
Al suo interno sono presenti due blocchi, la Control Unit e il Datapath.
Tutto ciò è conforme con quanto richiesto dalla consegna del progetto ovvero ingresso a 24 bit e uscita a 24 bit a prescindere
dal parallelismo adottato internamente per sviluppare i calcoli ed effettuare le numerose operazioni logiche.

\begin{figure}[H]
    \centering
    \includegraphics[width=0.75\textwidth]{Foto/Butterfly/Butterfly_schematic.jpg}
    \caption{Rappresentazione di una butterfly singola}
    \label{fig: Butterfly_schematic}
\end{figure}

\pagebreak