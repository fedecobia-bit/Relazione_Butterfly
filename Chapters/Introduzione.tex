\section{Introduzione}

\noindent La FFT (Fast Fourier Transform) è un'operazione fondamentale per tutti i sistemi di elaborazione dei segnali digitali.
È utilizzata nelle telecomunicazioni, nell'elaborazione audio e nei sistemi embedded ad alte prestazioni.
L'algoritmo FFT si basa sull'operatore butterfly, che è una struttura di manipolazione dei dati che esegue combinazioni lineari di dati complessi mediante somma, 
sottrazione e moltiplicazione con coefficienti anch'essi complessi.\par

Lo scopo di questo progetto è progettare un'unità di elaborazione dedicata per eseguire la FFT mediante la concatenazione di
più stati composti dalla singola Butterfly.
Si utilizzeranno tecniche di microprogrammazione e verranno considerati vincoli realistici dell'architettura hardware.
Più specificamente, questo progetto si occupa della gestione di dati complessi in una rappresentazione frazionaria a complemento
a due di 24 bit, dell'uso di un'aritmetica fixed point e dell'arrotondamento dei dati per tornare ad avere in uscita un parallelismo
di 24 bit.
Il lavoro include la derivazione del diagramma di flusso dei dati (DFD), l'ottimizzazione del datapath e dell'unità di controllo,
la completa descrizione dell'architettura in VHDL e la verifica funzionale attraverso simulazioni.
Infine, la Butterfly implementata deve essere utilizzata come blocco di base per l'implementazione e il collaudo di una FFT 16x16, 
che ne dimostra la validità e la scalabilità della soluzione.

\par\noindent Per creare la singola butterfly sono stati seguiti i seguenti passi:

\begin{itemize}
    \item Creazione del Data Flow Diagram
    \item Stima del tempo di vita delle variabili
    \item Creazione del Datapath
    \item Creazione della Control Unit (CU)
    \item Test finali
\end{itemize}

\noindent Data la necessità di utilizzare diversi blocchi logici quali moltiplicatori, sommatori, sottrattori, registri
e multiplexer sono state eseguite delle simulazioni intermedie rispetto ai punti appena descritti per facilitare il lavoro di debug.
Si è proceduto nel modo descritto in quanto è da preferire rispetto ad un approccio "trial and error" dove tutti i blocchi
non vengono testati e si procede solamente al test finale della Butterfly.
Nel caso fosse stata scelta questa strategia progettuale sarebbe stato pressoché impossibile andare a trovare dove fosse
l'errore nel caso si fosse verificato qualche malfunzionamento.\\

\pagebreak