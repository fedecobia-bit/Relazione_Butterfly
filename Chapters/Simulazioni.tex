\section{Simulazioni}

Le specifiche impongono due specifici modi di funzionamento della FFT: singola e continua.
Tutti i test riportati sotto sono stati presi dalla modalità continua ma questo non è ben visibile.
Per sopperire ad ogni possibile dubbio verranno riportate, prima dei test, due immagini che mostrano il sistema che lavora in 
modalità singola e continua.

\begin{figure}[H]
    \centering
    \includegraphics[width=1\textwidth]{Simulazioni/FFT/FFT_singola.png}
    \caption{Funzionamento in modalità singola}
    \label{fig: FFT_singola}
\end{figure}

\begin{figure}[H]
    \centering
    \includegraphics[width=1\textwidth]{Simulazioni/FFT/FFT_continua.png}
    \caption{Funzionamento in modalità continua}
    \label{fig: FFT_continua}
\end{figure}

\noindent A causa dell'aritmetica fixed point utilizzata nella FFT, alcuni campioni che dovrebbero essere zero presentano 
valori molto piccoli ma non nulli.
Questi valori sono dell'ordine di un LSB del formato Q1.23 (LSB = $1.19209 \cdot 10^{-7}$).
Sono causati dagli errori di quantizzazione e di arrotondamento che si verificano durante le operazioni di somma,
moltiplicazione e regolazione all'interno della Butterfly.
Queste componenti residue non hanno significato fisico né impatto sul contenuto spettrale del segnale
e possono essere considerate nulle entro la risoluzione numerica del sistema.

Oltre a questo si fa presente che i risultati riportati in \textit{Tab. \ref{tab: risultati_fft}} sono stati moltiplicati per 32
mentre le immagini presentano valori ancora da scalare.
Pertanto ogni valore letto nelle immagini va moltiplicato per 32 per ottenere il valore descritto nella tabella.

Infine, dalle figure sottostanti si può notare che alcuni campioni che escono dal blocco FFT iniziano a stabilizzarsi prima che
il segnale DONE venga asserito.
Tuttavia, per un corretto utilizzo dei dati, le uscite sono considerate valide e campionabili solo quando il segnale DONE è alto.
Il segnale DONE, quindi, funge da indicatore di validità per l'intero vettore di output, garantendo che tutti i passaggi
di calcolo interni e di propagazione dei dati siano completati.
Qualsiasi valore presente alle uscite prima dell'asserimento di DONE non deve essere utilizzato poiché potrebbero essere presenti
dei dati non corretti.

\begin{figure}[H]
    \centering
    \includegraphics[width=1\textwidth]{Simulazioni/FFT/vettore1_reale.png}
    \caption{Risultati FFT di V\textsubscript{1} - parte reale}
    \label{fig: vettore1_reale}
\end{figure}

\begin{figure}[H]
    \centering
    \includegraphics[width=1\textwidth]{Simulazioni/FFT/vettore1_immaginario.png}
    \caption{Risultati FFT di V\textsubscript{1} - parte immaginaria}
    \label{fig: vettore1_immaginario}
\end{figure}

\begin{figure}[H]
    \centering
    \includegraphics[width=1\textwidth]{Simulazioni/FFT/vettore2_reale.png}
    \caption{Risultati FFT di V\textsubscript{2} - parte reale}
    \label{fig: vettore2_reale}
\end{figure}

\begin{figure}[H]
    \centering
    \includegraphics[width=1\textwidth]{Simulazioni/FFT/vettore2_immaginario.png}
    \caption{Risultati FFT di V\textsubscript{2} - parte immaginaria}
    \label{fig: vettore2_immaginario}
\end{figure}

\begin{figure}[H]
    \centering
    \includegraphics[width=1\textwidth]{Simulazioni/FFT/vettore3_reale.png}
    \caption{Risultati FFT di V\textsubscript{3} - parte reale}
    \label{fig: vettore3_reale}
\end{figure}

\begin{figure}[H]
    \centering
    \includegraphics[width=1\textwidth]{Simulazioni/FFT/vettore3_immaginario.png}
    \caption{Risultati FFT di V\textsubscript{3} - parte immaginaria}
    \label{fig: vettore3_immaginario}
\end{figure}

\begin{figure}[H]
    \centering
    \includegraphics[width=1\textwidth]{Simulazioni/FFT/vettore4_reale.png}
    \caption{Risultati FFT di V\textsubscript{4} - parte reale}
    \label{fig: vettore4_reale}
\end{figure}

\begin{figure}[H]
    \centering
    \includegraphics[width=1\textwidth]{Simulazioni/FFT/vettore4_immaginario.png}
    \caption{Risultati FFT di V\textsubscript{4} - parte immaginaria}
    \label{fig: vettore4_immaginario}
\end{figure}

\begin{figure}[H]
    \centering
    \includegraphics[width=1\textwidth]{Simulazioni/FFT/vettore5_reale.png}
    \caption{Risultati FFT di V\textsubscript{5} - parte reale}
    \label{fig: vettore5_reale}
\end{figure}

\begin{figure}[H]
    \centering
    \includegraphics[width=1\textwidth]{Simulazioni/FFT/vettore5_immaginario.png}
    \caption{Risultati FFT di V\textsubscript{5} - parte immaginaria}
    \label{fig: vettore5_immaginario}
\end{figure}

\begin{figure}[H]
    \centering
    \includegraphics[width=1\textwidth]{Simulazioni/FFT/vettore6_reale.png}
    \caption{Risultati FFT di V\textsubscript{6} - parte reale}
    \label{fig: vettore6_reale}
\end{figure}

\begin{figure}[H]
    \centering
    \includegraphics[width=1\textwidth]{Simulazioni/FFT/vettore6_immaginario.png}
    \caption{Risultati FFT di V\textsubscript{6} - parte immaginaria}
    \label{fig: vettore6_immaginario}
\end{figure}

\pagebreak