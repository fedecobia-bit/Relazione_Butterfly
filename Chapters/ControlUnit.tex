\section{Control Unit}

\noindent La Control Unit è l'unità logica che decide le operazioni da far svolgere al datapath.
Come si vede da \textit{Fig. \ref{fig: control_unit}} nell'unità di controllo si distinguono tre componenti:
la late status PLA, la $\mu$ROM, e il datapath con il micro Instruction Register e il micro Address Register; questi verranno discussi in capitoli separati.


\subsection{Comandi e stati}

Gli stati della control unit sono stati codificati in modo da trarre il maggiore beneficio dalla tecnica del LATE STATUS:
gli indirizzi di stati che si susseguono sono distanti di solo un LSB. 

Gli stati di IDLE e DONE sono caratterizzati da LSB = 0 e CC\_Validation = 0: la late status pla usa questa informazione e determina il LSB dello stato successivo solamente in base al segnale di START,
in modo da ripetere lo stato di IDLE se è negato o di passare allo stato di START se è asserito. Quando la butterfly ha terminato un'esecuzione passa allo stato di DONE e poi in IDLE se non riceve 
un segnale di START. Lo stato di IDLE si ripete finché non viene ricevuto il segnale di START.

Lo stato di START è caratterizzato da LSB = 1 e CC\_Validation = 1: il LSB dello stato successivo è determinato dal segnale di SF\_2H\_1L, che viene asserito o negato dall'esterno un colpo di clock dopo il
segnale di START; in questo stato vengono anche campionati A, B e W.

Lo stato successivo gestisce i segnali di controllo per lo svolgimento della prima operazione, Br*Wr. Se nello stato precedente SF\_2H\_1L è stato asserito, 
viene anche operato un flip flop di tipo T per asserire il segnale di shift per il blocco rounding.

Gli stati successivi gestiscono i segnali per le altre operazioni di moltiplicazione, addizione, sottrazione e rounding.
Le codifiche degli stati e le istruzioni sono riportate in \textit{Fig. \ref{fig: stati_control_unit}}.

\begin{figure}[h!]
    \centering
    \includegraphics[width=\textwidth]{Foto/ControlUnit/CU_stati.png}
    \caption{Stati del sistema}
    \label{fig: stati_control_unit}
\end{figure}




\subsection{Struttura dell'unità di controllo}

\subsubsection{Late Status PLA}

\noindent Come da specifiche di progetto l'unità di controllo fa uso di un indirizzamento esplicito e della tecnica "Late Status".
Per velocizzare il sistema ed evitare un calo delle prestazioni la macchina può saltare il ritardo dovuto al campionamento del LSB da parte del $\mu$AR
se il resto dell'indirizzo rimane uguale, ovvero se un salto non avviene, ciò che discrimina la necessità di effettuare o no un salto è il bit meno significativo dell'indirizzo stesso.
La $\mu$ROM è stata suddivisa in locazioni pari e dispari (vedasi paragrafo "$\mu$ROM" per maggiore dettagli).
In \textit{Fig. \ref{fig: control_unit}} è possibile vedere le porte logiche che compongono la Status PLA e, riportati sotto la 
figura, le espressioni logiche che determinano il valore dei bit in uscita da questo blocco. La truth table della Late Status PLA è riportata in \textit{Tab. \ref{tab: comandi_control_unit}}.

\begin{table}[h!]
    \centering
    \begin{tabular}{|c|c|c|c||c|c|}
        \hline
        CC & LSB & START & SF\_2H\_1L & LSB\_OUT & CC\_OUT \\ \hline
         0 &  0  &   0   &     0      &     0    &    1    \\ \hline
         0 &  0  &   0   &     1      &     0    &    1    \\ \hline
         0 &  0  &   1   &     0      &     1    &    1    \\ \hline
         0 &  0  &   1   &     1      &     1    &    1    \\ \hline
         0 &  1  &   0   &     0      &     0    &    0    \\ \hline
         0 &  1  &   0   &     1      &     0    &    0    \\ \hline
         0 &  1  &   1   &     0      &     0    &    0    \\ \hline
         0 &  1  &   1   &     1      &     0    &    0    \\ \hline
         1 &  0  &   0   &     0      &     1    &    1    \\ \hline
         1 &  0  &   0   &     1      &     1    &    1    \\ \hline
         1 &  0  &   1   &     0      &     1    &    1    \\ \hline
         1 &  0  &   1   &     1      &     1    &    1    \\ \hline
         1 &  1  &   0   &     0      &     0    &    0    \\ \hline
         1 &  1  &   0   &     1      &     1    &    0    \\ \hline
         1 &  1  &   1   &     0      &     0    &    0    \\ \hline
         1 &  1  &   1   &     1      &     1    &    0    \\ \hline
    \end{tabular}
    \caption{Truth Table Late Status PLA}
    \label{tab: comandi_control_unit}
\end{table}

\subsubsection{uROM}

\noindent Come spiegato nel paragrafo precedente la $\mu$ROM è stata suddivisa in indirizzi pari e dispari. I 3 MSB del micro Address Register entrano 
sia nella parte pari che nella parte dispari, e la $\mu$ROM porta in uscita entrambi i dati su 22 bit.
Un MUX posto in uscita alla $\mu$ROM determina quale dei due lati deve essere campionato dal micro Instruction Register, usando come segnale di select
o il LSB del micro Address Register in caso di salto o il LSB collegato direttamente all'uscita della Status PLA.

\subsubsection{datapath, uAR e uIR}

\noindent 
il $\mu$IR è un registro a 22 bit, contiene un bit di CC\_Validation usato dalla Late Status PLA per determinare il prossimo LSB, 17 bit di segnali di controllo per il datapath, e 4 bit di indirizzo 
per il prossimo stato di cui il LSB viene passato alla Late Status PLA e i restanti bit al $\mu$AR.
Per evitare ritardi eccessivi, è stato deciso di collegare al $\mu$AR il clock senza alterazioni e al $\mu$IR il clock negato. Le micro istruzioni raggiungono il datapath della FFT 
prima del successivo fronte positivo del clock. 
Per implementare la tecnica del Late Status è stato aggiunto un multiplexer a 1 bit che dà il segnale di select al MUX in uscita alla $\mu$ROM.

\begin{figure}[H]
    \centering
    \includegraphics[width=0.6\textwidth]{Foto/ControlUnit/control_unit.jpg}
    \caption{Schema della CU implementato}
    \label{fig: control_unit}
\end{figure}

\pagebreak