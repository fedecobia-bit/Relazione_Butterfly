\section{Control Unit}

\noindent La Control Unit è l'unità logica che decide le operazioni da far svolgere al datapath.
Come si vede da \textit{Fig. \ref{fig: control_unit}} l'intera unità di controllo può essere suddivisa in tre parti:
la status PLA, la ROM e un datapath in miniatura utilizzato per far muovere i segnali all'interno del sistema.
Data l'importanza di questi tre macro blocchi verranno dedicati successivamente tre paragrafi per la descrizione dettagliata
di quest'ultimi.

Si vuole poi porre enfasi sulla stati e sui comandi utilizzati.
La scelta della codifica degli stati e dei comandi da utilizzare all'interno della butterfly è fondamentale per un corretto
funzionamento del sistema.
Per questo motivo si dedicherà un capitolo specifico dove verranno riportati i comandi e gli stati.

\subsection{Comandi e stati}

\begin{table}[h!]
    \centering
    \begin{tabular}{|c|c|c|}
        \hline
        STATO & CC\_VALIDATION & INDIRIZZO\\ \hline
        IDLE & 0 & 0000\\ \hline
        START & 1 & 0001\\ \hline
        M\textsubscript{1}, SH\textsubscript{0} & 0 & 0010\\ \hline
        M\textsubscript{1}, SH\textsubscript{1} & 0 & 0011\\ \hline
        M\textsubscript{2} & 1 & 0100\\ \hline
        M\textsubscript{3} & 0 & 0101\\ \hline
        M\textsubscript{4}, S\textsubscript{1} & 1 & 0110\\ \hline
        S\textsubscript{2} & 0 & 0111\\ \hline
        M\textsubscript{5}, D\textsubscript{1} & 1 & 1000\\ \hline
        M\textsubscript{6}, S\textsubscript{3} & 0 & 1001\\ \hline
        D\textsubscript{2}, SH\textsubscript{1} & 1 & 1010\\ \hline
        D\textsubscript{3}, SH\textsubscript{2} & 0 & 1011\\ \hline
        SH\textsubscript{3} & 1 & 1100\\ \hline
        SH\textsubscript{4} & 0 & 1101\\ \hline
        DONE & 0 & 1110\\ \hline
    \end{tabular}
    \caption{Stati del sistema}
    \label{tab: stati_control_unit}
\end{table}

\begin{table}[h!]
    \centering
    \begin{tabular}{|c|c|c|c|c|c|}
        \hline
        CC & LSB & START & SF\_2H\_1L & LSB\_OUT & CC\_OUT \\ \hline
         0 &  0  &   0   &     0      &     0    &    1    \\ \hline
         0 &  0  &   0   &     1      &     0    &    1    \\ \hline
         0 &  0  &   1   &     0      &     1    &    1    \\ \hline
         0 &  0  &   1   &     1      &     1    &    1    \\ \hline
         0 &  1  &   0   &     0      &     0    &    0    \\ \hline
         0 &  1  &   0   &     1      &     0    &    0    \\ \hline
         0 &  1  &   1   &     0      &     0    &    0    \\ \hline
         0 &  1  &   1   &     1      &     0    &    0    \\ \hline
         1 &  0  &   0   &     0      &     1    &    1    \\ \hline
         1 &  0  &   0   &     1      &     1    &    1    \\ \hline
         1 &  0  &   1   &     0      &     1    &    1    \\ \hline
         1 &  0  &   1   &     1      &     1    &    1    \\ \hline
         1 &  1  &   0   &     0      &     0    &    0    \\ \hline
         1 &  1  &   0   &     1      &     1    &    0    \\ \hline
         1 &  1  &   1   &     0      &     0    &    0    \\ \hline
         1 &  1  &   1   &     1      &     1    &    0    \\ \hline
    \end{tabular}
    \caption{Comandi}
    \label{tab: comandi_control_unit}
\end{table}


\subsection{Struttura dell'unità di controllo}

\subsubsection{Status PLA}

\noindent Come da specifiche di progetto l'unità di controllo fa uso di un indirizzamento esplicito e della tecnica "Late Status".
Per velocizzare il sistema ed evitare un calo delle prestazioni la macchina può saltare da un indirizzo all'altro, ciò che 
discrimina la necessità di effettuare o no un salto è il bit meno significativo dell'indirizzo stesso.
Per facilitare questi salti la ROM è stata suddivisa in locazioni pari e dispari (vedasi paragrafo "ROM" per maggiore dettagli).
In \textit{Fig. \ref{fig: control_unit}} è possibile vedere le porte logiche che compongono la Status PLA e, riportati sotto la 
figura, le espressioni logiche che determinano il valore dei bit in uscita da questo blocco.

\subsubsection{ROM}

\noindent Come spiegato nel paragrafo precedente la ROM è stata suddivisa in indirizzi pari e dispari.
Questo serve per regolare/facilitare i vari salti che può essere necessario compiere durante lo svolgimento delle operazioni.
Si fa notare al lettore che dalla ROM esce contemporaneamente sia un il dato memorizzato in un indirizzo pari che uno memorizzato
in un indirizzo dispari.
Ciò che effettivamente arriva in uscita e su cui vengono fatte le successive considerazioni viene selezionato da un MUX posto
in uscita alla ROM.
Facendo in questo modo evitiamo di dover aspettare tutto il tempo necessario all'accesso in memoria.

\subsubsection{datapath}

\noindent Questo "datapath" non è da intendere come il datapath della butterfly spiegato nel relativo capitolo.
In questo caso si vuole intendere solamente quell'insieme di collegamenti che permettono il corretto flusso dei dati che devono
transitare all'interno dell'unità di controllo.
Banalmente, serviranno una serie di bus per collegare le varie parti della Control Unit tra di loro e con il resto della butterfly
per ottenere il corretto funzionamento del sistema.
Si è voluto specificare quanto appena detto poichè nell'Appendice (\textit{List. \ref{listing: control_unit_datapath}}) è presente
del codice chiamato "control unit datapath" e si voleva fare un distinguo tra quello e il datapath vero e proprio.

\begin{figure}[H]
    \centering
    \includegraphics[width=0.75\textwidth]{Foto/ControlUnit/control_unit.jpg}
    \caption{Schema della CU implementato}
    \label{fig: control_unit}
\end{figure}

\pagebreak