\section{Butterfly e FFT}

\subsection{Butterfly}

\noindent Dopo aver descritto, tramite linguaggio VHDL, i vari blocchi precedentemente descritti si è passati alla creazione di un
blocco butterfly singolo per un test intermedio e per accertarsi che tutto funzionasse in modo sinergico.
I listati possono essere consultati nel \textit{Cap. \ref{cap: Appendice}}.

Questo test è risultato importante perchè ha permesso di correggere alcuni piccoli errori che non erano stati individuati durante
i test precedenti.
Grazie a queste modifiche il singolo blocco butterfly ha funzionato correttamente durante i successivi test e ciò a permesso di 
proseguire nell'implementazione della FFT 16x16 senza doversi preoccupare di possibili errori commessi in precedenza.\\

\noindent Come si vede da \textit{Fig. \ref{fig: Butterfly_schematic}} la butterfly singola riceve in ingresso segnali con
parallelismo 24 bit e il segnale SF\_2H\_1L per gestire lo shift.
In uscita presenta un bus con parallelismo 24 bit.
Al suo interno sono presenti due blocchi, la Control Unit e il Datapath.
Tutto ciò è conforme con quanto richiesto dalla consegna del progetto ovvero ingresso a 24 bit e uscita a 24 bit a prescindere
dal parallelismo adottato internamente per sviluppare i calcoli ed effettuare le numerose operazioni logiche.

\begin{figure}[H]
    \centering
    \includegraphics[width=0.75\textwidth]{Foto/Butterfly/Butterfly_schematic.jpg}
    \caption{Rappresentazione di una butterfly singola}
    \label{fig: Butterfly_schematic}
\end{figure}

\subsection{FFT}

\noindent Una volta creato la singola butterfly si è proceduto con la creazione della struttura che permette l'implementazione
hardware della FFT.
Per fare ciò la singola butterfly è stata replicata trentadue volte.
Volendo rendere più semplice la gestione dei segnali in ciascuna butterfly sono stati assegnati staticamente i valori di SF\_2H\_1L,
Wr e Wi.

\[
\begin{aligned}
\mathbf{V}_1 &=
\left[
\begin{array}{r}
-1 \\ -1 \\ -1 \\ -1 \\
-1 \\ -1 \\ -1 \\ -1 \\
-1 \\ -1 \\ -1 \\ -1 \\
-1 \\ -1 \\ -1 \\ -1
\end{array}
\right]
\qquad
FFT(\mathbf{V}_1) &=
\left[
\begin{array}{c}
-16 \\ 0 \\ 0 \\ 0 \\
0 \\ 0 \\ 0 \\ 0 \\
0 \\ 0 \\ 0 \\ 0 \\
0 \\ 0 \\ 0 \\ 0
\end{array}
\right]
\qquad
\mathbf{V}_2 &=
\left[
\begin{array}{r}
-1 \\  0 \\  1 \\  0 \\
-1 \\  0 \\  1 \\  0 \\
-1 \\  0 \\  1 \\  0 \\
-1 \\  0 \\  1 \\  0
\end{array}
\right]
\qquad
FFT(\mathbf{V}_2) &=
\left[
\begin{array}{c}
0 \\  0 \\  0 \\  0 \\
-8 \\  0 \\  0 \\  0 \\
0 \\  0 \\  0 \\  0 \\
-8 \\  0 \\  0 \\  0
\end{array}
\right]
\\[1em] % serve per andare a capo
%\qquad % se si va a capo commentare questo
\mathbf{V}_3 &=
\left[
\begin{array}{r}
 1 \\ 0 \\ 0 \\ 0 \\
 0 \\ 0 \\ 0 \\ 0 \\
 0 \\ 0 \\ 0 \\ 0 \\
 0 \\ 0 \\ 0 \\ 0
\end{array}
\right]
\qquad
FFT(\mathbf{V}_3) &=
\left[
\begin{array}{r}
 1 \\ 1 \\ 1 \\ 1 \\
 1 \\ 1 \\ 1 \\ 1 \\
 1 \\ 1 \\ 1 \\ 1 \\
 1 \\ 1 \\ 1 \\ 1
\end{array}
\right]
\qquad
\mathbf{V}_4 &=
\left[
\begin{array}{r}
-1 \\ -1 \\  1 \\  1 \\
-1 \\ -1 \\  1 \\  1 \\
-1 \\ -1 \\  1 \\  1 \\
-1 \\ -1 \\  1 \\  1
\end{array}
\right]
\qquad
FFT(\mathbf{V}_4) &=
\left[
\begin{array}{r}
-1 \\ -1 \\  1 \\  1 \\
-1 \\ -1 \\  1 \\  1 \\
-1 \\ -1 \\  1 \\  1 \\
-1 \\ -1 \\  1 \\  1
\end{array}
\right]
\\[1em] % serve per andare a capo
%\qquad % se si va a capo commentare questo
\mathbf{V}_5 &=
\left[
\begin{array}{r}
 0.5 \\  0.5 \\  0.5 \\  0.5 \\
 0.5 \\  0.5 \\  0.5 \\  0.5 \\
 0.5 \\ -0.5 \\ -0.5 \\ -0.5 \\
-0.5 \\ -0.5 \\ -0.5 \\ -0.5
\end{array}
\right]
\qquad
FFT(\mathbf{V}_5) &=
\left[
\begin{array}{r}
 ? \\  ? \\  ? \\  ? \\
 ? \\  ? \\  ? \\  ? \\
 ? \\  ? \\  ? \\  ? \\
 ? \\  ? \\  ? \\  ? 
\end{array}
\right]
\qquad
\mathbf{V}_6 &=
\left[
\begin{array}{r}
0 \\ 0 \\ 0 \\ 0 \\
0 \\ 0 \\ 0 \\ 0 \\
0.75 \\ 0 \\ 0 \\ 0 \\
0 \\ 0 \\ 0 \\ 0
\end{array}
\right]
\qquad
FFT(\mathbf{V}_6) &=
\left[
\begin{array}{r}
-0.75 \\ 0.75 \\ -0.75 \\ 0.75 \\
-0.75 \\ 0.75 \\ -0.75 \\ 0.75 \\
-0.75 \\ 0.75 \\ -0.75 \\ 0.75 \\
-0.75 \\ 0.75 \\ -0.75 \\ 0.75 \\
\end{array}
\right]
\end{aligned}
\]



\pagebreak