\section{Datapath}

\noindent Dopo aver stimato e valutato il tempo di vita delle variabili si è iniziato a progettare il datapath necessario a
svolgere tutte le operazioni richieste della CU.
Il primo datapath studiato è rappresentato in \textit{Fig. \ref{fig: datapath_originale}}.
Come si vede dallo schema non è stato apportato ancora nessun miglioramento volto all'ottimizzazione del numero di BUS e/o al
loro parallelismo.

Successivamente si è preso come riferimento il Data Flow Diagram (DFD) e si sono apportate delle migliorie per ciò che concerne
l'efficienza dello schema circuitale.
Come si vede da \textit{Fig. \ref{fig: datapath}} il register file è stato mantenuto e tutti i segnali che devono entrare all'interno
del Datapath passano attraverso di esso.
A valle sono stati inseriti dei MUX con lo scopo di selezionare i vari dati da mandare ai blocchi logici.
Dal DFD è chiaramente visibile il fatto che i vari segnali, durante il loro tempo di vita, entreranno solo in specifici blocchi
logici, risulta perciò superfluo e deleterio avere dei collegamenti (BUS) tra ogni uscita del register file e ogni blocco logico.
Dato che l'uscita di un blocco logico potrebbe dover essere riutilizzata in uno step successivo si è fatto uso di registri intermedi
che permettono di memorizzare e di riportare il dato in ingresso quando risulta necessario.
Ciò è conveniente in quanto, facendo in questo modo, si risparmiano molte scritture su BUS che risultano essere lente e dispendiose
in termini energetici.
Infine, è stato esplicitato il blocco ROM Rounding che serve per arrotondare.

\begin{figure}[H]
    \centering
    \includegraphics[width=0.75\textwidth]{Foto/Datapath/datapath_originale.jpg}
    \caption{Schema del datapath iniziale}
    \label{fig: datapath_originale}
\end{figure}

\begin{figure}[H]
    \centering
    \includegraphics[width=0.75\textwidth]{Foto/Datapath/datapath.jpg}
    \caption{Schema del datapath finale}
    \label{fig: datapath}
\end{figure}

\subsection{ROM Rounding}

\noindent Come richiesto nella consegna del progetto per avere un uscita nel formato Q1.23 è necessario fare uso del ROM rounding.
Questa tecnica consiste nell'arrotondare gli N bit in ingresso al blocco arrotondatore con una look-up-table salvata all'interna
di una memoria ROM.
Per leggere i dati presenti all'interno della memoria sarà necessario fornire all'ingresso un indirizzo.
La scelta del numero di bit dell'indirizzo, e conseguentemente del numero di celle della memoria, è una scelta critica per il
progetto in quanto un indirizzo di pochi bit consente di avere una memoria piccola (e che quindi richiede poco spazio su Silicio)
ma permette un arrotondamento peggio.
Nel caso sia necessario ottenere un arrotondamento più preciso, e quindi con errore minore, allora risulta obbligatorio aumentare
il numero di bit di indirizzo per permettere l'indirizzamento di più celle di memoria.
Considerando che si volevano salvare 5 bit per riga è stato scelto come numero di bit per l'indirizzo 5.
Si riporta di seguito una tabella che mette in relazione il numero di bit dell'indirizzo con il numero di righe dell ROM e con il
numero di bit totali da memorizzare.

\begin{table}[h!]
    \centering
    \begin{tabular}{|c|c|c|}
        \hline
        bit indirizzo  & righe ROM & bit totali \\ \hline
        3 & 8   & 40\\ \hline
        5 & 32  & 160\\ \hline
        7 & 128 & 640\\ \hline
        9 & 512 & 2560\\ \hline
    \end{tabular}
    \caption{Relazione tra il numero di bit di indirizzo della ROM, il numero di righe ed il numero totale di bit memorizzati}
    \label{tab: indirizzo_righe_bit}
\end{table}

\noindent Bisogna anche considerare il bias e l'errore medio.
Avendo come specifica di progetto l'utilizzo del metodo "Round to Nearest Even" si è dovuto scegliere un indirizzo composto da un
numero di bit della mantissa e bit di scarto disposti in maniera tale da minimizzare sia il bias che l'errore.
Di seguito si riportano i test effettuati:

\begin{table}[h!]
    \centering
    \begin{tabular}{|c|c|c|}
        \hline
        bit indirizzo  & bias & errore \\ \hline
        3 &   & \\ \hline
        5 &   & \\ \hline
        7 &   & \\ \hline
        9 &   & \\ \hline
    \end{tabular}
    \caption{Relazione tra il numero di bit di indirizzo della ROM, il bias e l'errore}
    \label{tab: indirizzo_bias_errore}
\end{table}

\noindent Dopo aver considerato tutte le opzioni, sia dal lato di area occupata che dal lato bias/errore, è stato scelto di comporre
l'indirizzo della ROM con gli ultimi 3 bit della mantissa (LSB mantissa) e con i primi 2 bit dello scarto (MSB scarto).
Si riporta in \textit{Fig. \ref{fig: ROM_rounding}} lo schema del ROM rounding implementato.
Si può apprezzare la presenza della ROM, un registro posto in ingresso e uno in uscita usati per rendere i dati disponibili sul
fronte del clock dato che la ROM è puramente combinatoria e, infine, il parallelismo dei bus espresso col numero di fianco al bus
stesso.

\begin{figure}[H]
    \centering
    \includegraphics[width=0.75\textwidth]{Foto/Datapath/ROM_rounding.jpg}
    \caption{Schema del ROM rounding implementato}
    \label{fig: ROM_rounding}
\end{figure}

\pagebreak