\section{Datapath}

\noindent Dopo aver stimato e valutato il tempo di vita delle variabili si è iniziato a progettare il datapath necessario a
svolgere tutte le operazioni richieste della CU.
Il primo datapath studiato è rappresentato in \textit{Fig. \ref{fig: datapath_originale}}.
Come si vede dallo schema non è stato apportato ancora nessun miglioramento volto all'ottimizzazione del numero di BUS e/o al
loro parallelismo.

Successivamente si è preso come riferimento il Data Flow Diagram (DFD) e si sono apportate delle migliorie per ciò che concerne
l'efficienza dello schema circuitale.
Come si vede da \textit{Fig. \ref{fig: datapath}} il register file è stato mantenuto e tutti i segnali che devono entrare all'interno
del Datapath passano attraverso di esso.
A valle sono stati inseriti dei MUX con lo scopo di selezionare i vari dati da mandare ai blocchi logici.
Dal DFD è chiaramente visibile il fatto che i vari segnali, durante il loro tempo di vita, entreranno solo in specifici blocchi
logici, risulta perciò superfluo e deleterio avere dei collegamenti (BUS) tra ogni uscita del register file e ogni blocco logico.
Dao che l'uscita di un blocco logico potrebbe dover essere riutilizzata in uno step successivo

\begin{figure}[H]
    \centering
    \includegraphics[width=0.75\textwidth]{Foto/Datapath/datapath_originale.jpg}
    \caption{Schema del datapath iniziale}
    \label{fig: datapath_originale}
\end{figure}

\begin{figure}[H]
    \centering
    \includegraphics[width=0.75\textwidth]{Foto/Datapath/datapath.jpg}
    \caption{Schema del datapath finale}
    \label{fig: datapath}
\end{figure}