% Gestione avanzata delle immagini e testo contornato
\usepackage{graphicx,wrapfig}

% Supporto per la lingua italiana e sillabazione
\usepackage[italian]{babel}

% Formule matematiche avanzate e simboli extra
\usepackage{amsmath}
\usepackage{amssymb}

% Creazione di link ipertestuali e URL cliccabili
\usepackage{hyperref}

% Configurazione dei margini e del layout della pagina
\usepackage{geometry}

% Gestione di celle su più righe nelle tabelle
\usepackage{multirow}

% Codifica dei caratteri in entrata (UTF-8)
\usepackage[utf8]{inputenc}

% Codifica dei font per la corretta stampa dei caratteri accentati
\usepackage[T1]{fontenc}

% Inserimento di testo "così com'è" (codice o testo piano)
\usepackage{verbatim}

% Inserimento di un'immagine nel frontespizio (titolo)
\usepackage[cc]{titlepic}

% Macro semplificate per formule di fisica (derivate, vettori, ecc.)
\usepackage{physics}

% Personalizzazione delle didascalie e gestione sotto-figure
\usepackage{caption}
\usepackage{subcaption}

% Personalizzazione dell'Indice e delle liste di figure/tabelle
\usepackage[subfigure]{tocloft}

% Creazione di box colorati, incorniciati e decorati
\usepackage[most]{tcolorbox}

% Gestione avanzata della posizione delle note a piè di pagina
\usepackage[bottom]{footmisc}

% Estensioni per l'allineamento e il formato delle tabelle
\usepackage{array}

% Controllo preciso del posizionamento degli oggetti flottanti
\usepackage{float}

% Inserimento di codice sorgente con formattazione specifica
\usepackage{listings}

% Gestione dei colori per testo, sfondi e box
\usepackage{xcolor}