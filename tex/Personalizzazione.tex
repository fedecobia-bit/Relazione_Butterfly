\geometry{
    a4paper,
    top = 2cm,
    bottom = 2cm,
    left = 1.5cm,
    right = 1.5cm,
    marginparwidth = 1cm
}

\lstdefinelanguage{VHDL}{
  keywords={entity, architecture, begin, end, signal, process, if, then, else,
            elsif, case, when, others, port, map, generic, in, out, std_logic,
            std_logic_vector,
            library, use, entity, architecture, is, begin, end,
            generic, port, map, component, configuration,
            signal, variable, constant, subtype, type, file,
            integer, natural, positive,
            real, time, boolean, bit, bit_vector,
            std_logic, std_logic_vector,
            signed, unsigned,
            in, out, inout, buffer,
            process, wait, until, loop, for, while, exit, next,
            if, then, else, elsif,
            case, when, others,
            assert, report, severity,
            rising_edge, falling_edge,
            and, or, nand, nor, xor, xnor, not,
            abs, mod, rem,
            sll, srl, sla, sra, rol, ror,
            to_integer, to_unsigned, to_signed,
            resize, std_logic_arith, numeric_std,
            event, stable, quiet, transaction,
            last_event, last_value, driving, active,
            generate, for, if, label,
            open, range, downto, to,
            alias, shared, disconnect, guarded,
            transport, inertial, reject
  },
  sensitive=true,
  comment=[l]--,
  morecomment=[s]{/*}{*/},
  morestring=[b]"
}
\lstset{
  language=VHDL,
  basicstyle=\ttfamily\small,
  keywordstyle=\color{blue},
  commentstyle=\color{gray},
  stringstyle=\color{red},
  numbers=left,
  numberstyle=\tiny,
  stepnumber=1,
  frame=single,
  breaklines=true,
  captionpos=b
}

% Permette di inserire numeri romani nel testo -> \rom{3} = III
\makeatletter
\newcommand*{\rom}[1]{\expandafter\@slowromancap\romannumeral #1@}
\makeatother

% le subsubsections nell'indice le rende le lettere dell'alfabeto
\renewcommand{\thesubsubsection}{\alph{subsubsection}.}
\setcounter{secnumdepth}{3}

% Per usarlo \begin(myquote)
\newtcolorbox{myquote}[1][]{%
    colback=black!5,
    colframe=black!5,
    notitle,
    sharp corners,
    borderline west={2pt}{0pt}{red!80!black},
    enhanced,
    breakable,
}