\documentclass{article}
% Per inserire delle immagini
\usepackage{graphicx,wrapfig}
% Language setting
\usepackage[italian]{babel}
% per inserire i simboli matematici
\usepackage{amsmath}
\usepackage{amssymb}
% Per includere degli eventuali link
\usepackage{hyperref}
\hypersetup{
    colorlinks = true, 
    linkcolor = black,
    urlcolor = blue
    %allcolors = black
}
% Impostazioni della pagina
\usepackage{geometry}
\geometry{
    a4paper,
    top = 2cm,
    bottom = 2cm,
    left = 1.5cm,
    right = 1.5cm,
    marginparwidth = 1cm
}
% Roba
\usepackage{multirow}
\usepackage[utf8]{inputenc}
\usepackage{verbatim}
\usepackage[cc]{titlepic}
\usepackage{physics}
\usepackage{caption}
\usepackage{subcaption}
\usepackage[subfigure]{tocloft}
\usepackage[most]{tcolorbox}
\usepackage[bottom]{footmisc}
\usepackage{array}
\usepackage{float}

\makeatletter
\newcommand*{\numeroromano}[1]{\expandafter\@slowromancap\romannumeral #1@}
\makeatother

\title{
    \textbf{Politecnico di Torino}\\
    Scuola di Ingegneria e Architettura\\
    \bigskip\bigskip
    \textbf{Optoelettronica}\\
    Prof. Mariangela Gioannini\\
    Prof. Lorenzo Luigi Columbo\\
    \bigskip\bigskip\bigskip\bigskip
    \textbf{Relazione laboratori}\\
    \bigskip\bigskip
    \vfill
    %\vspace{3cm}
\begin{figure}[htp]
    \centering
    \includegraphics[scale=0.5]{Style/logo.png}
    %\vspace{-10pt}
\end{figure}
%\vspace{5cm}
\vfill
}


\author{
    Federico Cobianchi\\
    \bigskip
}
\date{A.A. 2025/2026}
\pagebreak
\begin{document}
\maketitle
\pagenumbering{Roman}
\setcounter{page}{1}

% Permette di inserire numeri romani nel testo -> \rom{3} = III
\makeatletter
\newcommand*{\rom}[1]{\expandafter\@slowromancap\romannumeral #1@}
\makeatother

% le subsubsections nell'indice le rende le lettere dell'alfabeto
\renewcommand{\thesubsubsection}{\alph{subsubsection}.}
\setcounter{secnumdepth}{3}

% Per usarlo \begin(myquote)
\newtcolorbox{myquote}[1][]{%
    colback=black!5,
    colframe=black!5,
    notitle,
    sharp corners,
    borderline west={2pt}{0pt}{red!80!black},
    enhanced,
    breakable,
}

% Crea l'indice
\newpage
\renewcommand{\contentsname}{Indice}
\tableofcontents % Creazione indice
\setcounter{page}{2}
% Inizio del vero e proprio documento
\newpage
\pagenumbering{arabic}
\setcounter{page}{1}

\section{Introduzione}

\noindent La FFT (Fast Fourier Transform) è un'operazione fondamentale per tutti i sistemi di elaborazione dei segnali digitali.
È utilizzata nelle telecomunicazioni, nell'elaborazione audio e nei sistemi embedded ad alte prestazioni.
L'algoritmo FFT si basa sull'operazione butterfly, che è una struttura di manipolazione dei dati che esegue combinazioni lineari di dati complessi mediante somma, 
sottrazione e moltiplicazione con coefficienti complessi.\par

Lo scopo di questo progetto è progettare un'unità di elaborazione dedicata per eseguire la Butterfly FFT, utilizzando tecniche di microprogrammazione e considerando
vincoli realistici dell'architettura hardware. Più specificamente, questo progetto si occupa della gestione di dati complessi in una rappresentazione frazionaria a complemento a due di
24 bit, dell'uso della Scansione in Virgola Mobile a Blocco Incondizionata per gestire il sovraccarico e dell'implementazione di un datapath ottimizzato dati
i vincoli di risorse computazionali limitate e pipeline interna.
Il lavoro include la derivazione del diagramma di flusso dei dati dell'algoritmo, l'ottimizzazione del datapath e dell'unità di controllo, la completa descrizione dell'architettura
in VHDL e la verifica funzionale attraverso simulazioni. Infine, la Butterfly implementata deve essere utilizzata come blocco di base per l'implementazione e il collaudo di una FFT 16x16, 
che ne dimostra la validità e la scalabilità della soluzione.

\par\noindent Per creare la singola butterfly sono stati seguiti i seguenti passi:

\begin{itemize}
    \item Creazione del Data Flow Diagram
    \item Stima del tempo di vita delle variabili
    \item Creazione del Datapath
    \item Creazione della Control Unit (CU)
    \item Test finali
\end{itemize}

\noindent Data la necessità di utilizzare diversi blocchi logici quali moltiplicatori, sommatori, sottrattori, registri e multiplexer sono state eseguite delle simulazioni intermedie rispetto
ai punti appena descritti per facilitare il lavoro di debug.
Si è proceduto nel modo descritto i quanto è da preferire rispetto ad un approccio "trial and error" dove tutti i blocchi non vengono testati e si procede solamente al test finale della butterfly.
Nel caso fosse stata scelta questa strategia progettuale sarebbe stato pressoché impossibile andare a trovare dove fosse l'errore nel caso si fosse verificato qualche malfunzionamento.\\

\noindent Una volta completata la singola butterfly è stato creato il processore che esegue la FFT unendo tra loro le varie unità necessarie per adempiere alla richiesta finale del progetto.
Una volta implementato il tutto il sistema è stato testato nella sua interezza per constatare l'effettivo funzionamento.

\pagebreak
\section{Data Flow Diagram}
\noindent In questo capitolo si parlerà delle specifiche imposte sui blocchi logici.
Si andranno a confrontare gli approcci "As Soon As Possible" \(ASAP\) e "As Late As Possible" \(ALAP\).
Infine verrà illustrato l'approccio che è stato utilizzato per ottimizzare le tempistiche dell'algoritmo.

\subsection{Specifiche sui blocchi operazionali}

\noindent In questo Progetto si supponeva di poter utilizzare per ciascuna Butterfly un unico blocco moltiplicatore.
Questo blocco è in grado di poter svolgere sia la moltiplicazione tra due numeri in ingresso sia lo shift (moltiplicazione per 2).
Le due operazioni sono selezionabili attraverso un segnale di controllo esterno al blocco operatore che verrà fornito dalla
Contro Unit (CU).
Si è supposto che il moltiplicatore avesse due livelli di pipeline, ovvero che il risultato fosse disponibile al registro
di uscita dopo 3 colpi di clock.
L'operazione di shift, al contrario, ha un singolo livello di pipeline, dunque l'uscita è disponibile dopo 2 colpi di clock.
Il blocco moltiplicatore è stato rappresentato in \textit{Fig. \ref{fig: blockdataflowdiagram}} con il blocco \(verde\) e
l'operazione di shift è stata rappresentata con il blocco \(viola\).

Si è supposto di avere a disposizione un singolo elemento per le operazioni di somma e uno per le sottrazioni.
I blocchi sommatore e sottrattore hanno ciascuno un livello di pipeline e sono rappresentati rispettivamente
dal blocco \(rosso\) e \(blu\).

Al fine di rappresentare anche l'operazione di ROM Rounding presente alla fine dell'algoritmo è stato deciso di impiegare
un colpo di clock per l'operazione di arrotondamento (blocco \(azzurro\)) e utilizzare successivamente un registro
controllato esternamente per mantenere in vita le variabili di uscita della butterfly.

\begin{figure}[H]
    \centering
    \includegraphics[width=0.75\textwidth]{Foto/DataFlowDiagram/BlocchiDataFlowDiagram.jpg}
    \caption{Blocchi elementari del data flow diagram}
    \label{fig: blockdataflowdiagram}
\end{figure}

\subsection{Approccio ASAP}
\noindent L'approccio "As Soon As Possible", è un approccio che predilige lo svolgimento delle operazioni non appena si
ha disponibilità di blocchi logici.
Come si può notare in \textit{Fig. \ref{fig: asapdataflow}} sarebbero necessari 6 blocchi moltiplicatori e 2 blocchi
sommatore e sottrattore.
Andando a confrontare le richieste di questa metodologia con le specifiche di progetto risulta subito evidente che questo approccio
non è applicabile.

\begin{figure}[H]
    \centering
    \includegraphics[width=0.75\textwidth]{Foto/DataFlowDiagram/asap_DataFlow.jpg}
    \caption{Data Flow Diagram con pproccio ASAP}
    \label{fig: asapdataflow}
\end{figure}

\subsection{Approccio ALAP}
\noindent Questo approccio predilige lo svolgimento delle operazioni il più tardi possibile.
Lo schema riportato in \textit{Fig. \ref{fig: alapdataflow}} mostra come il numero di blocchi operazionali richiesti sia
inferiore rispetto all'approccio ASAP.
Vengono infatti utilizzati 2 elementi per ciascuna operazione di moltiplicazione, somma, differenza e arrotondamento.
Anche in questo caso però non viene rispettato il limite numerico di 1 blocco moltiplicatore per Butterfly.
\'{E} stato dunque studiato un approccio che ci permettesse di rimanere entro le specifiche richieste dal progetto.


\begin{figure}[H]
    \centering
    \includegraphics[width=0.75\textwidth]{Foto/DataFlowDiagram/alap_DataFlow.jpg}
    \caption{Data Flow Diagram con approccio ALAP}
    \label{fig: alapdataflow}
\end{figure}

\subsection{Approccio scelto}
\noindent Volendo ottimizzare al massimo il sistema pur rimanendo all'interno delle specifiche imposte si è optato per 
il Data Flow Diagram illustrato in \textit{Fig. \ref{fig: dataflowdiagram}}.
Questo approccio è stato studiato per l'esecuzione dell'algoritmo in modo da avere ad ogni stadio un unico blocco logico 
per tipo di operazione.

\begin{figure}[H]
    \centering
    \includegraphics[width=0.75\textwidth]{Foto/DataFlowDiagram/DataFlowDiagram.jpg}
    \caption{Data flow diagram ottimizzato}
    \label{fig: dataflowdiagram}
\end{figure}

\subsection{Tempo di vita delle variabili}
\noindent In \textit{Fig. \ref{fig: tempodivita}} viene illustrato il tempo di vita delle variabili derivato dal Data Flow Diagram
che è stato scelto.
Questo studio risulta utile per capire quanto una variabile deve rimanere in vita e, di conseguenza, quanti registri sono necessari
per gestire tutte le variabili all'interno del sistema.
Andando a studiare qual è il numero massimo richiesto di registri si riduce il costo e l'area occupata su Silicio.
Inoltre, si facilita lo studio di quali e quanti segnali sono richiesti dalla CU per controllare il flusso dei dati.

In questo caso però, per come è stato progettato il sistema, gli unici registri che necessitano un segnale di controllo
esterno sono i registri di ingresso e d'uscita della Butterfly.
I registri posizionati tra i blocchi logici, dato che sono "vivi" per un solo colpo di clock, non necessitano di segnali di
controllo.

\begin{figure}[H]
    \centering
    \includegraphics[width=0.75\textwidth]{Foto/DataFlowDiagram/tempo_di_vita_variabili.jpg}
    \caption{Tempo di vita delle variabili}
    \label{fig: tempodivita}
\end{figure}

\pagebreak
\section{Datapath}

\noindent Dopo aver stimato e valutato il tempo di vita delle variabili si è iniziato a progettare il datapath necessario a
svolgere tutte le operazioni richieste della CU.
Il primo datapath studiato è rappresentato in \textit{Fig. \ref{fig: datapath_originale}}.
Come si vede dallo schema non è stato apportato ancora nessun miglioramento volto all'ottimizzazione del numero di BUS e/o al
loro parallelismo.

Successivamente si è preso come riferimento il Data Flow Diagram (DFD) e si sono apportate delle migliorie per ciò che concerne
l'efficienza dello schema circuitale.
Come si vede da \textit{Fig. \ref{fig: datapath}} il register file è stato mantenuto e tutti i segnali che devono entrare all'interno
del Datapath passano attraverso di esso.
A valle sono stati inseriti dei MUX con lo scopo di selezionare i vari dati da mandare ai blocchi logici.
Dal DFD è chiaramente visibile il fatto che i vari segnali, durante il loro tempo di vita, entreranno solo in specifici blocchi
logici, risulta perciò superfluo e deleterio avere dei collegamenti (BUS) tra ogni uscita del register file e ogni blocco logico.
Dato che l'uscita di un blocco logico potrebbe dover essere riutilizzata in uno step successivo si è fatto uso di registri intermedi
che permettono di memorizzare e di riportare il dato in ingresso quando risulta necessario.
Ciò è conveniente in quanto, facendo in questo modo, si risparmiano molte scritture su BUS che risultano essere lente e dispendiose
in termini energetici.
Infine, è stato esplicitato il blocco ROM Rounding che serve per arrotondare.

\begin{figure}[H]
    \centering
    \includegraphics[width=0.75\textwidth]{Foto/Datapath/datapath_originale.jpg}
    \caption{Schema del datapath iniziale}
    \label{fig: datapath_originale}
\end{figure}

\begin{figure}[H]
    \centering
    \includegraphics[width=0.75\textwidth]{Foto/Datapath/datapath.jpg}
    \caption{Schema del datapath finale}
    \label{fig: datapath}
\end{figure}

\subsection{ROM Rounding}

\noindent Come richiesto nella consegna del progetto per avere un uscita nel formato Q1.23 è necessario fare uso del ROM rounding.
Questa tecnica consiste nell'arrotondare gli N bit in ingresso al blocco arrotondatore con una look-up-table salvata all'interna
di una memoria ROM.
Per leggere i dati presenti all'interno della memoria sarà necessario fornire all'ingresso un indirizzo.
La scelta del numero di bit dell'indirizzo, e conseguentemente del numero di celle della memoria, è una scelta critica per il
progetto in quanto un indirizzo di pochi bit consente di avere una memoria piccola (e che quindi richiede poco spazio su Silicio)
ma permette un arrotondamento peggio.
Nel caso sia necessario ottenere un arrotondamento più preciso, e quindi con errore minore, allora risulta obbligatorio aumentare
il numero di bit di indirizzo per permettere l'indirizzamento di più celle di memoria.
Considerando che si volevano salvare 5 bit per riga è stato scelto come numero di bit per l'indirizzo 5.
Si riporta di seguito una tabella che mette in relazione il numero di bit dell'indirizzo con il numero di righe dell ROM e con il
numero di bit totali da memorizzare.

\begin{table}[h!]
    \centering
    \begin{tabular}{|c|c|c|}
        \hline
        bit indirizzo  & righe ROM & bit totali \\ \hline
        3 & 8   & 40\\ \hline
        5 & 32  & 160\\ \hline
        7 & 128 & 640\\ \hline
        9 & 512 & 2560\\ \hline
    \end{tabular}
    \caption{Relazione tra il numero di bit di indirizzo della ROM, il numero di righe ed il numero totale di bit memorizzati}
    \label{tab: indirizzo_righe_bit}
\end{table}

\noindent Bisogna anche considerare il bias e l'errore medio.
Avendo come specifica di progetto l'utilizzo del metodo "Round to Nearest Even" si è dovuto scegliere un indirizzo composto da un
numero di bit della mantissa e bit di scarto disposti in maniera tale da minimizzare sia il bias che l'errore.
Di seguito si riportano i test effettuati:

\begin{table}[h!]
    \centering
    \begin{tabular}{|c|c|c|}
        \hline
        bit indirizzo  & bias & errore \\ \hline
        3 &   & \\ \hline
        5 &   & \\ \hline
        7 &   & \\ \hline
        9 &   & \\ \hline
    \end{tabular}
    \caption{Relazione tra il numero di bit di indirizzo della ROM, il bias e l'errore}
    \label{tab: indirizzo_bias_errore}
\end{table}

\noindent Dopo aver considerato tutte le opzioni, sia dal lato di area occupata che dal lato bias/errore, è stato scelto di comporre
l'indirizzo della ROM con gli ultimi 3 bit della mantissa (LSB mantissa) e con i primi 2 bit dello scarto (MSB scarto).
Si riporta in \textit{Fig. \ref{fig: ROM_rounding}} lo schema del ROM rounding implementato.
Si può apprezzare la presenza della ROM, un registro posto in ingresso e uno in uscita usati per rendere i dati disponibili sul
fronte del clock dato che la ROM è puramente combinatoria e, infine, il parallelismo dei bus espresso col numero di fianco al bus
stesso.

\begin{figure}[H]
    \centering
    \includegraphics[width=0.75\textwidth]{Foto/Datapath/ROM_rounding.jpg}
    \caption{Schema del ROM rounding implementato}
    \label{fig: ROM_rounding}
\end{figure}

\pagebreak

% \bibliographystyle{ieeetr}
% \bibliography{sample}
% \addcontentsline{toc}{section}{References}

% Comandi da usare per sincronizzare con github
%
% git status
% git add .
% git commit -m "descrizione sensata"
% git push

\end{document}